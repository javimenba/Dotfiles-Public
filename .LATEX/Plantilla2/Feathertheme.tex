\documentclass[10pt]{beamer}
\usetheme[
%%% option passed to the outer theme
%    progressstyle=fixedCircCnt,   % fixedCircCnt, movingCircCnt (moving is deault)
  ]{Feather}
  
% If you want to change the colors of the various elements in the theme, edit and uncomment the following lines

% Change the bar colors:
%\setbeamercolor{Feather}{fg=red!20,bg=red}

% Change the color of the structural elements:
%\setbeamercolor{structure}{fg=red}

% Change the frame title text color:
%\setbeamercolor{frametitle}{fg=blue}

% Change the normal text color background:
%\setbeamercolor{normal text}{fg=black,bg=gray!10}

%-------------------------------------------------------
% INCLUDE PACKAGES
%-------------------------------------------------------

\usepackage[utf8]{inputenc}
\usepackage[english]{babel}
\usepackage[T1]{fontenc}
\usepackage{helvet}
\usepackage[document]{ragged2e}
\usepackage{graphicx}
\usepackage[rightcaption]{sidecap}
\usepackage[none]{hyphenat}
\usepackage{animate,media9,movie15}
\usepackage[ ]{hyperref}

%-------------------------------------------------------
% DEFFINING AND REDEFINING COMMANDS
%-------------------------------------------------------

% colored hyperlinks
\newcommand{\chref}[2]{
  \href{#1}{{\usebeamercolor[bg]{Feather}#2}}
}

%-------------------------------------------------------
% INFORMATION IN THE TITLE PAGE
%-------------------------------------------------------

\title[] % [] is optional - is placed on the bottom of the sidebar on every slide
{ % is placed on the title page
      \textbf{Ventanas Inteligentes}
}

\subtitle[Ventanas Inteligentes]
{
      \textbf{}
}

\author[Teoria Electromagnetica]
{      Francisco Javier MendozaBautista
	{\ttfamily }
}

\institute[]
{
     Universidad de Guanajuato \\Ingeniería Electrónica\\
  
  %there must be an empty line above this line - otherwise some unwanted space is added between the university and the country (I do not know why;( )
}

\date{\today}

%-------------------------------------------------------
% THE BODY OF THE PRESENTATION
%-------------------------------------------------------

\begin{document}

%-------------------------------------------------------
% THE TITLEPAGE
%-------------------------------------------------------

{\1% % this is the name of the PDF file for the background
\begin{frame}[plain,noframenumbering] % the plain option removes the header from the title page, noframenumbering removes the numbering of this frame only
  \titlepage % call the title page information from above
\end{frame}}


\begin{frame}{Contenido}{}
\tableofcontents
\end{frame}

%-------------------------------------------------------
\section{Ventanas inteligentes}
%-------------------------------------------------------
\subsection{¿Que son las ventanas inteligentes?}
\begin{frame}{Ventanas inteligentes}{¿Que son las ventanas inteligentes?}

\begin{columns}
	\column{0.5\textwidth}
	\justify
	Se trata de unas ventanas que tienen incorporado un sistema que en cuestión de segundos, mediante un interruptor, se puede activar una tecnología que provoca unas reacciones químicas y físicas haciendo que el vidrio transparente se convierta en opaco
	\column{0.5\textwidth}
    \begin{figure}
    	\includegraphics[scale=0.5]{Imagenes/1}
    	\caption{Ventanas inteligentes}
    \end{figure}
\end{columns}
\end{frame}







\section{Demostración}
\subsection{Video}
\begin{frame}{Demostración}{Video}
   



\begin{figure}		
   		\includemovie[
   		poster,
   		autoplay,
   	    externalviewer,
   		inline=false,
   		text={\small(click)}
   		]{6cm}{6cm}{video.mp4}
   		
   		%    \flashmovie{Wildlife.wmv}
   	\end{figure}
 

\end{frame}




\section{Bibliografía}
\begin{frame}{Bibliografía}
          	\begin{thebibliography}{1}
          	
          	
          	\bibitem{IEEEhowto:kopka}
          	 Construcción 21 España, Ventanas inteligentes 
          	 \textcolor{blue}{\url{www.construction21.org/espana/articles/es/que-son-las-ventanas-inteligentes.html}}
          	
          \end{thebibliography}
\end{frame}


\end{document}